\documentclass[10pt,letterpaper]{article}

%%%%%%%%%%%%%%%%%%%%%%%%%%%%%%%%%%%%%%%%%%%%%%%%%%%%%%%%%%%%%%%%%%%%%%%
% Packages to include
%%%%%%%%%%%%%%%%%%%%%%%%%%%%%%%%%%%%%%%%%%%%%%%%%%%%%%%%%%%%%%%%%%%%%%%
\usepackage{multind}        % For adding multiple indexes
\usepackage{multicol}       % For making more than two columns of index
\usepackage{fancyhdr}
\usepackage{setspace}       % allow the use of double spacing
\usepackage{graphicx,color}
\usepackage{colortbl}       % Provides \rowcolor for colored table cells
\usepackage{textcomp}       % allows the use of special symbols
\usepackage{array}          % allows the use of different column types in tables
\usepackage{longtable}      % allows tables that span multiple pages
\usepackage{verbatim}       % allow the use of \verbatiminput
\usepackage{amssymb}        % Allows use of special math symbols, like \square
\usepackage{amsmath}        % Allows use of \text to get text in math mode
\usepackage{gensymb}        % Allows use of \celsius symbol
\usepackage{multirow}       % Allows for table data to span multiple rows
\usepackage{keystroke}      % For illustrating front panel keys
\usepackage{calc}           % For doing math with dimensions
\usepackage{subfigure}      % For adding figures within figures
\usepackage{rotating}       % For sidewaystable and sidewaysfigure
\usepackage{wrapfig}        % For placing figures at the side of a page
\usepackage{boxedminipage}  % Just what it sounds like
\usepackage[subfigure]{tocloft}    % For formatting table of contents
\usepackage{hhline}         % For more flexible lines in tables
\usepackage{capt-of}        % Allows captions not just for floats
\usepackage{nbaseprt}       % For formatting numbers using \nbaseprint
\usepackage{varioref}       % Allows page references like "this page"
\usepackage{listings}       % Provides \lstlisting for code formatting
\usepackage[obeyspaces]{url}% Provides \path for verbatim with line breaks
\usepackage{dirtree} % For drawing file directory trees
\usepackage{textcomp}       % For trademark symbol
\usepackage{xcolor}         % For colored text

% Intelligently insert space after trademark symbols
\usepackage{xspace}

% Allows verbatim input inside boxes
\usepackage{fancyvrb}

% Allows strikeout text.
\usepackage[normalem]{ulem}

% Allows typesetting SI units
\usepackage{siunitx}

%%%%%%%%%%%%%%%%%%%%%%%%%%%%%% Caption %%%%%%%%%%%%%%%%%%%%%%%%%%%%%%%
%
% Provides control of caption spacing in tables and figures
\usepackage[tableposition=below]{caption}
% Caption spacing for longtables
\captionsetup[longtable]{skip=1em}



%%%%%%%%%%%%%%%%%%%%%%%%%%%%%% Hyperref %%%%%%%%%%%%%%%%%%%%%%%%%%%%%%
\usepackage[colorlinks = true,
            linkcolor = blue,
            urlcolor = blue]{hyperref}

%%%%%%%%%%%%%%%%%%%%%%%%%%%%%%%%%%%%%%%%%%%%%%%%%%%%%%%%%%%%%%%%%%%%%%%
% Geometry package to simplify defining margins and paper
%
% The dvips driver option writes the paper size to a variable read by xdvi.
% Options can be set with arguments to \usepackage or the \geometry macro
%%%%%%%%%%%%%%%%%%%%%%%%%%%%%%%%%%%%%%%%%%%%%%%%%%%%%%%%%%%%%%%%%%%%%%%
\usepackage[dvips]{geometry}        %Lets xdvi figure out paper size
    \geometry{paper=letterpaper}    %Sets letter size
    \geometry{margin=1in}       %1 inch margins all around

% --------------------------------------------------------------------
% Define lengths
% --------------------------------------------------------------------
%\setlength{\topmargin}{-.1in}
\setlength{\headheight}{.2in}%page header height
\setlength{\headsep}{.25in}%space between header and text
\setlength{\footskip}{.7in}
%\setlength{\textheight}{8.5in}
\renewcommand{\textfraction}{0.05}  %minimum fraction of page for text
\renewcommand{\topfraction}{0.95}   %max fraction of page for floats at top
\renewcommand{\bottomfraction}{0.95}    %max fraction of page for floats at bottom
% max fraction of floatpage that should have floats
\renewcommand{\floatpagefraction}{0.95}

% \tabindent
% Add indentation for items in a table.  I sometimes need this when trying
% to make an itemized list where each subitem is inside its own table cell
\newcommand{\tabindent}{\hspace*{5mm}}

\newlength{\pad}
\setlength{\pad}{.5cm}

% Minimum height of integer parameter boxes in tables.  Setting this at
% the beginning of a table provides a way to keep number boxes the same
% sizes while the parboxes they label change heights.
\newlength{\paramheight}
\setlength{\paramheight}{1cm}

% The height of numbers (or characters without newlines).  Use this to
% calculate offsets for centering numbers or characters in boxes.
\newlength{\numheight}

% Width of "trap" label for extra information warnings
\newlength{\warnwidth}
\setlength{\warnwidth}{2cm}

% Width of sideby figures describing front panel features
\newlength{\sidefigwidth}
\setlength{\sidefigwidth}{6cm}

% Width of text for sideby figures
\newlength{\sidetextwidth}
\setlength{\sidetextwidth}{9cm}

\setlength{\unitlength}{1cm}

% Width of column containing remote commands
\newlength{\cmdwidth}
\setlength{\cmdwidth}{4cm}

% \cftsubsecnumwidth
% Set the space between the beginning of the subsection number and the
% beginning of the subsection title in the toc
\setlength{\cftsubsecnumwidth}{1cm}

% \cftsubsubsecnumwidth
% Set the space between the beginning of the subsubsection number and the
% beginning of the subsubsection title in the toc
\setlength{\cftsubsubsecnumwidth}{1.5cm}

% \specitemwidth
% Set the space for the attribute column in the spec pages, where the
% specifications are given as attribute <space> value
\newcommand{\specitemwidth}{3in}

% --------------------------------------------------------------------
% Define colors
% --------------------------------------------------------------------

\definecolor{codegray}{gray}{0.9} % Shading for codeboxes

%%%%%%%%%%%%%%%%%%%%%%%%%%%%%%%%%%%%%%%%%%%%%%%%%%%%%%%%%%%%%%%%%%%%%%%%%%%
%Set counters
%%%%%%%%%%%%%%%%%%%%%%%%%%%%%%%%%%%%%%%%%%%%%%%%%%%%%%%%%%%%%%%%%%%%%%%%%%%
\setcounter{totalnumber}{5} % Maximum number of floats on a page

%%%%%%%%%%%%%%%%%%%%%%%%% Fancy page styles %%%%%%%%%%%%%%%%%%%%%%%%%%

% normal page style
\fancypagestyle{normal}{
    % The rules at the page top and bottom
    \renewcommand{\headrulewidth}{0.4pt}
    \renewcommand{\footrulewidth}{0.4pt}
    \lhead{}
    \rhead{\nouppercase \rightmark}
    \rfoot{\normalrightfoot{}}
    \cfoot{\thepage{}}
    \lfoot{\normalleftfoot{}}
} % End of normal page definition

%*********************************************************
%\lfoot{kick me}
%\rfoot{\today}
%\cfoot{\thepage}
%\renewcommand{\headrulewidth}{0.4pt}
%\renewcommand{\footrulewidth}{0.4pt}
%\setlength{\headrulewidth}{1pt}%set these widths to zero to make them gone
%\setlength{\footrulewidth}{1pt}

\fancypagestyle{toc}{
    \lhead{Contents}
    \rhead{}
    \fancyfoot[C]{\thepage}
    \renewcommand{\headrulewidth}{0.4pt}
    \renewcommand{\footrulewidth}{0.4pt}
}%End of toc fancy page style definition

\fancypagestyle{references}{
    \lhead{References}
    \rhead{}
    \fancyfoot[C]{\thepage}
    \renewcommand{\headrulewidth}{0.4pt}
    \renewcommand{\footrulewidth}{0.4pt}
}%End of references fancy page style definition

%%%%%%%%%%%%%%%%%%%%%%%%%%%%%%%%%%%%%%%%%%%%%%%%%%%%%%%%%%%%%%%%%%%%%%%%%%%
%Format footnotes
%%%%%%%%%%%%%%%%%%%%%%%%%%%%%%%%%%%%%%%%%%%%%%%%%%%%%%%%%%%%%%%%%%%%%%%%%%%
\renewcommand{\thefootnote}{\arabic{footnote}}
%\renewcommand{\thefootnote}{\fnsymbol{footnote}}   %For symbols
\renewcommand{\thempfootnote}{\arabic{mpfootnote}}  %For footnotes in minipages

%%%%%%%%%%%%%%%%%%%%%%%%%%%%%%%%%%%%%%%%%%%%%%%%%%%%%%%%%%%%%%%%%%%%%%%%%%%
%The bibliography style
%%%%%%%%%%%%%%%%%%%%%%%%%%%%%%%%%%%%%%%%%%%%%%%%%%%%%%%%%%%%%%%%%%%%%%%%%%%
\bibliographystyle{doctools/latex/IEEEtran}
