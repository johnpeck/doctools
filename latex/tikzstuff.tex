\usepackage{tikz} % For making flow charts
\usetikzlibrary{shapes,decorations,arrows}
\usetikzlibrary{positioning}

%---------------------------Define nodes and lines---------------------
\tikzstyle{funcnode} = [draw=black, inner sep=1pt]
\newlength{\fnodewidth} % Width of the box for funcnode
\setlength{\fnodewidth}{6cm}

\tikzstyle{arrow} = [draw, -latex]

% The decision node.
% See the xcolor package documentation for help defining fill colors.
% The draw option causes a border to be drawn.
% It's better to set node distance here than in the picture.
\tikzstyle{decision} = [diamond, draw, fill=blue!10, text width=2cm,
    align=center, node distance=4cm, inner sep=2pt]

% The block node.  Use this for flowcharts.
\tikzstyle{block} = [rectangle, draw, fill=blue!10, text width=2cm,
    align=center, rounded corners=2pt, minimum height=4em,
    node distance=3cm]
    
% Generic annotation
\tikzstyle{ann} = [draw=none,fill=none,right]

%---------------------------End nodes and lines------------------------

% funcnode
% Create a node that describes a function.  The function name is printed
% on a shaded header, in boldface.
%
% This only works inside a tikzpicture environment.  This looks wrong
% when using \doublespacing -- the shaded region is too thick.
%
% Usage: \funcnode{<position>,<separation>}{<node name>}{<function name}
%                    {<node text contents>};
%
% See the sample_tikz.tex file for an example of all this.
\newcommand{\funcnode}[4]{
    \node [funcnode, #1] (#2) {
        \begin{minipage}{\fnodewidth}
            \begin{tabular}{l}
                \rowcolor[gray]{0.8}
                \begin{minipage}[c]{\textwidth - 2\tabcolsep}
                    \singlespacing
                    \begin{center}
                        \vspace{-\baselineskip}
                        \textbf{#3}
                    \end{center}
                \end{minipage}\\
                \begin{minipage}[c]{\textwidth - 2\tabcolsep}
                    \singlespacing
                    \begin{minipage}[c]{\textwidth - 2\tabcolsep}
                    #4
                    \end{minipage}
                    \vspace{0.5\baselineskip}\\
                \end{minipage}\\
            \end{tabular}
        \end{minipage}
    } % end node
} % end funcnode
