\documentclass[10pt,letterpaper]{article}

%%%%%%%%%%%%%%%%%%%%%%%%%%%%%%%%%%%%%%%%%%%%%%%%%%%%%%%%%%%%%%%%%%%%
%Packages to include
%%%%%%%%%%%%%%%%%%%%%%%%%%%%%%%%%%%%%%%%%%%%%%%%%%%%%%%%%%%%%%%%%%%%
\usepackage{fancyhdr}
%\usepackage{pslatex}       %use postscript fonts by default
\usepackage{fullpage}       %use 1 inch margins all around
\usepackage{setspace}       %allow the use of double spacing
\usepackage[dvips]{graphicx}
\usepackage{textcomp}       %allows the use of special symbols
\usepackage{array}      %allows the use of different column types in tables
\usepackage{longtable}      %allows tables that span multiple pages
\usepackage{verbatim}       %allow the use of \verbatiminput
\usepackage{amssymb}        %Allows use of special math symbols, like \square
\usepackage{amsmath}        %Allows use of \text to get text in math mode
\usepackage{gensymb}        %Allows use of \celsius symbol
\usepackage{multirow}       %Allows for table data to span multiple rows
\usepackage{moreverb}       %Makes including Matlab code easier
\usepackage{keystroke}      %For illustrating front panel keys
\usepackage{calc}       %For doing math with dimensions
%For timing symbols like falling and rising edges
\usepackage[electronic,geometry]{ifsym}
\usepackage{subfigure}      %For adding figures within figures
\usepackage{rotating}       %For sidewaystable and sidewaysfigure
\usepackage{wrapfig}        %For placing figures at the side of a page
\usepackage{lgrind}     %For including scheme code
\usepackage{pst-gantt}  %For gantt charts
\usepackage{xr}             %For cross-references across documents
\usepackage{color}
\usepackage{colortbl}   %For shading elements in tables
\usepackage{hhline}     %Alternative to cline
\usepackage{ifthen}     %For conditionals
\usepackage{appendix}   %For Peter Wilson's appendix package
\usepackage{boxedminipage}  %Just what it sounds like
\usepackage{enumitem}   % For making un-indented lists
\usepackage[subfigure]{tocloft}    % For formatting table of contents
\usepackage{nbaseprt}  % For formatting hexadecimal numbers

%%%%%%%%%%%%%%%%%%%%%%%%%%%%%%%%%%%%%%%%%%%%%%%%%%%%%%%%%%%%%%%%%%%%%
%Define lengths
%%%%%%%%%%%%%%%%%%%%%%%%%%%%%%%%%%%%%%%%%%%%%%%%%%%%%%%%%%%%%%%%%%%%%
\special{papersize=8.5in,11in}  %Pass letter paper size to xdvi
\setlength{\topmargin}{-.1in}
\setlength{\headheight}{.2in}%page header height
\setlength{\headsep}{.25in}%space between header and text
\setlength{\footskip}{.7in}
\setlength{\textheight}{8.5in}
\renewcommand{\textfraction}{0.05}  %minimum fraction of page for text
\renewcommand{\topfraction}{0.95}   %max fraction of page for floats at top
\renewcommand{\bottomfraction}{0.95}    %max fraction of page for floats at bottom
\renewcommand{\floatpagefraction}{0.95} %max fraction of floatpage that should have floats
\newcommand{\vs}{\vspace{.2cm}}
\newcommand{\hs}{\hspace{.1cm}}
% Add indentation for items in a table.  I sometimes need this when trying
% to make an itemized list where each subitem is inside its own table cell
\newcommand{\tabindent}{\hspace*{3mm}}




%%%%%%%%%%%%%%%%%%%%%%%%%%%%%%%%%%%%%%%%%%%%%%%%%%%%%%%%%%%%%%%%%%%%%%%%%%%
%Set counters
%%%%%%%%%%%%%%%%%%%%%%%%%%%%%%%%%%%%%%%%%%%%%%%%%%%%%%%%%%%%%%%%%%%%%%%%%%%
\setcounter{totalnumber}{5}%Maximum number of floats on a page


%%%%%%%%%%%%%%%%%%%%%%%%%%%%%%%%%%%%%%%%%%%%%%%%%%%%%%%%%%%%%%%%%%%%%%%%%%%
%Define the fancy page style
%%%%%%%%%%%%%%%%%%%%%%%%%%%%%%%%%%%%%%%%%%%%%%%%%%%%%%%%%%%%%%%%%%%%%%%%%%%
\fancypagestyle{normal}{
    \lhead{\nouppercase \leftmark}
    \rhead{\nouppercase \rightmark}
    \rfoot{\today}
    \cfoot{\thepage}
    \renewcommand{\headrulewidth}{0.4pt}
    \renewcommand{\footrulewidth}{0.4pt}
}%End of the normal page style definition
%*********uncomment this to put the SRS logo on************
%\lfoot{
%\begin{minipage}[t]{3in}
%\scalebox{1}
%{
%\includegraphics*{srs_long.ps}
%}
%\end{minipage}
%}
%*********************************************************
%\lfoot{kick me}
%\rfoot{\today}
%\cfoot{\thepage}
%\renewcommand{\headrulewidth}{0.4pt}
%\renewcommand{\footrulewidth}{0.4pt}
%\setlength{\headrulewidth}{1pt}%set these widths to zero to make them gone
%\setlength{\footrulewidth}{1pt}

\fancypagestyle{toc}{
    \lhead{Contents}
    \rhead{}
    \fancyfoot[C]{\thepage}
    \renewcommand{\headrulewidth}{0.4pt}
    \renewcommand{\footrulewidth}{0.4pt}
}%End of toc fancy page style definition

\fancypagestyle{references}{
    \lhead{References}
    \rhead{}
    \fancyfoot[C]{\thepage}
    \renewcommand{\headrulewidth}{0.4pt}
    \renewcommand{\footrulewidth}{0.4pt}
}%End of references fancy page style definition


%----------------------------------------------------------------------
% Configure section numbers
%
% The code word that prints the current value of the counter in your 
% document is \thecounter. That is, \thepart, \thechapter, \thesection 
% and so on.
%----------------------------------------------------------------------

% Redefine the section numbers to reference the part number.  
% For example, the first section in the first part will look like
% I.1 instead of just 1.
\renewcommand{\thesection}{\thepart.\arabic{section}}

%----------------------------------------------------------------------
% Configure the table of contents
%
% See the tocloft package documentation for descriptions of the \cft
% commands (contents, figures, tables)
%----------------------------------------------------------------------

% These next commands control the size of the box in which the title
% number is formatted.  The title number will be flushleft inside of
% the box.  Note that this box does not grow as the title number grows.
% So big title numbers like 1.2.1234 will need bigger boxes, and you'll
% need to set these by hand.  This is actually a feature -- when you
% look at the toc, all subsubsection titles will be indented the same
% amount, regardless of the width of the title number.

% \cftsecnumwidth -- the width from the beginning of the title number
% to the title when the number terminates in a section number
\setlength{\cftsecnumwidth}{1cm}

% \cftsubsecnumwidth -- the width from the beginning of the title number
% to the title when the number terminates in a subsection number
\setlength{\cftsubsecnumwidth}{1.5cm}

% \cftsubsubsecnumwidth -- the width from the beginning of the title 
% number to the title when the number terminates in a subsubsection number
\setlength{\cftsubsubsecnumwidth}{2cm}




%%%%%%%%%%%%%%%%%%%%%%%%%%%%%%%%%%%%%%%%%%%%%%%%%%%%%%%%%%%%%%%%%%%%%%%%%%%
%Format footnotes
%%%%%%%%%%%%%%%%%%%%%%%%%%%%%%%%%%%%%%%%%%%%%%%%%%%%%%%%%%%%%%%%%%%%%%%%%%%
\renewcommand{\thefootnote}{\arabic{footnote}}
%\renewcommand{\thefootnote}{\fnsymbol{footnote}}   %For symbols
\renewcommand{\thempfootnote}{\arabic{mpfootnote}}  %For footnotes in minipages

%%%%%%%%%%%%%%%%%%%%%%%%%%%%%%%%%%%%%%%%%%%%%%%%%%%%%%%%%%%%%%%%%%%%%%%
% Some useful document variables
%%%%%%%%%%%%%%%%%%%%%%%%%%%%%%%%%%%%%%%%%%%%%%%%%%%%%%%%%%%%%%%%%%%%%%%



%%%%%%%%%%%%%%%%%%%%%%%%%%%%%%%%%%%%%%%%%%%%%%%%%%%%%%%%%%%%%%%%%%%%%%%%%%%
%The bibliography style
%%%%%%%%%%%%%%%%%%%%%%%%%%%%%%%%%%%%%%%%%%%%%%%%%%%%%%%%%%%%%%%%%%%%%%%%%%%
\bibliographystyle{doctools/latex/IEEEtran}
